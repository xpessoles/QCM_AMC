\element{codeurs}{
\begin{question}{codeurs 01}
Soit un codeur mesurant la position d'un moteur, ayant une résolution de 48 tops/tours. Donner sa résolution en degrés.
\AMCOpen{lines=5}{\wrongchoice[F]{f}\scoring{0}\wrongchoice[P]{p}\scoring{1}\correctchoice[J]{j}\scoring{2}}

%\begin{multicols}{4}
%	\begin{reponses}	
%	\bonne{555}
%	\mauvaise{\lstinline{nombrenombrenombre}}
%	\mauvaise{\lstinline{15}}
%	\mauvaise{\lstinline{15.0}}
%	\end{reponses}
%	\end{multicols}
\end{question}
%%%%%%%
% 360/48 = 7,5 °
%%%%%%%




\begin{question}{codeurs 02}
Soit un codeur mesurant la position d'un moteur, ayant une résolution de 48 tops/tours.
Le moteur est suivi d'un réducteur de rapport 32. Donner la résolution en degrés au niveau de la sortie du réducteur.
\AMCOpen{lines=5}{\wrongchoice[F]{f}\scoring{0}\wrongchoice[P]{p}\scoring{1}\correctchoice[J]{j}\scoring{2}}
\end{question}
%%%%%%%
% 360/48/32 = 0,234 °
%%%%%%%


\begin{question}{codeurs 03}
Soit un codeur mesurant la position d'un moteur. Ce codeur est constitué d'un disque de 12 fentes, 2 canaux en quadrature. Donner la résolution en degrés au niveau de la sortie du moteur.
\AMCOpen{lines=5}{\wrongchoice[F]{f}\scoring{0}\wrongchoice[P]{p}\scoring{1}\correctchoice[J]{j}\scoring{2}}
\end{question}
%%%%%%%
% 12 x 2 x 2 = 48 infos par tour
% 360/48 = 7,5 °
%%%%%%%



\begin{question}{codeurs 04}
Soit un codeur mesurant la position d'un moteur. Ce codeur est constitué d'un disque de 12 fentes, 2 canaux en quadrature. Le moteur est suivi d'un réducteur de rapport 32. Donner la résolution en degrés au niveau de la sortie du réducteur.
\AMCOpen{lines=5}{\wrongchoice[F]{f}\scoring{0}\wrongchoice[P]{p}\scoring{1}\correctchoice[J]{j}\scoring{2}}
\end{question}
%%%%%%%
% 12 x 2 x 2 = 48 infos par tour
% 360/48/32 = 0,234 °
%%%%%%%


\begin{question}{codeurs 05}
Soit un codeur mesurant la position d'un moteur. La documentation stipule 500 impulsions, 3 canaux. Donner la résolution en degrés au niveau de la sortie du moteur.
\AMCOpen{lines=5}{\wrongchoice[F]{f}\scoring{0}\wrongchoice[P]{p}\scoring{1}\correctchoice[J]{j}\scoring{2}}
\end{question}
%%%
%2000 tops/tour
% 360/2000 = 0,18°
%%%

\begin{question}{codeurs 06}
Soit un codeur mesurant la position d'un moteur. La documentation stipule 500 impulsions, 3 canaux. Le moteur est suivi d'un réducteur de rapport 15,88. Donner la résolution en degrés au niveau de la sortie du réducteur.
\AMCOpen{lines=5}{\wrongchoice[F]{f}\scoring{0}\wrongchoice[P]{p}\scoring{1}\correctchoice[J]{j}\scoring{2}}
\end{question}
%%%
%2000 tops/tour
% 360/2000/15,88 = 0,011°
%%%


\begin{question}{codeurs 07}
Soit un codeur mesurant la position d'un moteur. La documentation stipule 1000 impulsions par tour.  Donner la résolution en degrés au niveau de la sortie du moteur.
\AMCOpen{lines=5}{\wrongchoice[F]{f}\scoring{0}\wrongchoice[P]{p}\scoring{1}\correctchoice[J]{j}\scoring{2}}
\end{question}
%%%
% 360/1000 = 0,36°
%%%

\begin{question}{codeurs 08}
Soit un codeur mesurant la position d'un moteur. La documentation stipule 1000 impulsions par tour.  Le moteur est suivi d'un réducteur de rapport 3. Donner la résolution en degrés au niveau de la sortie du réducteur.
\AMCOpen{lines=5}{\wrongchoice[F]{f}\scoring{0}\wrongchoice[P]{p}\scoring{1}\correctchoice[J]{j}\scoring{2}}
\end{question}
%%%
% 360/1000/3 = 0,12°
%%%


\begin{question}{codeurs 09}
Soit un codeur mesurant la position d'un moteur. La documentation stipule 1000 impulsions par tour.  Le moteur est suivi d'un réducteur de rapport 3. Le réducteur est suivi d'un système poulie-courroie (poulie de largeur \SI{25}{mm}, de pas \SI{5}{mm}, de 31 dents et de rayon \SI{24,67}{mm}). Donner la résolution en mm au niveau de la courroie.
\AMCOpen{lines=5}{\wrongchoice[F]{f}\scoring{0}\wrongchoice[P]{p}\scoring{1}\correctchoice[J]{j}\scoring{2}}
\end{question}
%%%
% 360/1000/3 = 0,12° = 0,002 rad
% 0.002*24.57 = 0,05 mm
%%%

%\begin{question}{codeurs 10}
%Soit un codeur mesurant la position d'un moteur. La documentation stipule 1000 impulsions par tour.  Le moteur est suivi d'un réducteur de rapport 3. Le réducteur est suivi d'un système poulie-courroie (poulie de largeur \SI{25}{mm}, de pas \SI{5}{mm}, de 31 dents et de rayon \SI{24,67}{mm}). Donner la résolution en mm au niveau de la courroie.
%\AMCOpen{lines=5}{\wrongchoice[F]{f}\scoring{0}\wrongchoice[P]{p}\scoring{1}\correctchoice[J]{j}\scoring{2}}
%\end{question}


\begin{question}{codeurs 11} 
Soit un codeur mesurant la position d'un moteur. La documentation stipule 1000 impulsions par tour.
Le vitesse maximale du moteur est de 5000 tour/min. Quelle doit être la fréquence minimale d'acquisition de la carte d'acquisition ?
\AMCOpen{lines=5}{\wrongchoice[F]{f}\scoring{0}\wrongchoice[P]{p}\scoring{1}\correctchoice[J]{j}\scoring{2}}
\end{question}
%%%
% 5000 tr/min = 83,333 tr/s
% 83 333 tops /s soit 83 kHz
%%%

\begin{question}{codeurs 12}
Soit un codeur mesurant la position d'un moteur. La documentation stipule 500 fentes 2 voies en quadrature.
Le vitesse maximale du moteur est de 8000 tour/min. Quelle doit être la fréquence minimale d'acquisition de la carte d'acquisition ?

\AMCOpen{lines=5}{\wrongchoice[F]{f}\scoring{0}\wrongchoice[P]{p}\scoring{1}\correctchoice[J]{j}\scoring{2}}
\end{question}
%%%
% 133,333 tr/s
% 2000 infos par tour
% 266 666 infos par s soit 266 kHz


}