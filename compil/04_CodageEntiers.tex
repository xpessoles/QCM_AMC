\element{CodageEntiers}{
\begin{question}{entiers 01} % genumsi nreveret
	Quel est l'entier positif codé en base 2 sur 8 bits par le code 0011 1010 ?
	\begin{multicols}{4}
	\begin{reponses}	
	\bonne{58}
	\mauvaise{45}
	\mauvaise{25}
	\mauvaise{-12}
	\end{reponses}
	\end{multicols}
\end{question}}

\element{CodageEntiers}{
\begin{question}{entiers 02} % genumsi nreveret
	Le résultat de l'addition des deux nombres binaires 1101 et 0101 est:
	\begin{multicols}{4}
	\begin{reponses}	
	\bonne{10010}
	\mauvaise{10110}
	\mauvaise{10011}
	\mauvaise{11010}
	\end{reponses}
	\end{multicols}
\end{question}}

\element{CodageEntiers}{
\begin{question}{entiers 03} % genumsi nreveret
	Convertir  la valeur décimale 155 en binaire (sur un octet).
	\begin{multicols}{4}
	\begin{reponses}	
	\bonne{10011011}
	\mauvaise{11011011}
	\mauvaise{01111111}
	\mauvaise{10010111}
	\end{reponses}
	\end{multicols}
\end{question}}

\element{CodageEntiers}{
\begin{question}{entiers 04} % genumsi OSUPK8
	Quelle est la valeur décimale de l'entier binaire 00011010 ?
	\begin{multicols}{4}
	\begin{reponses}	
	\bonne{26}
	\mauvaise{22}
	\mauvaise{51}
	\mauvaise{24}
	\end{reponses}
	\end{multicols}
\end{question}}

\element{CodageEntiers}{
\begin{question}{entiers 05} % genumsi OSUPK8
	Donner le résultat de l'addition binaire : 1101 + 1001.

	\begin{multicols}{4}
	\begin{reponses}	
	\bonne{10110}
	\mauvaise{01001}
	\mauvaise{00110}
	\mauvaise{11010}
	\end{reponses}
	\end{multicols}
\end{question}}

\element{CodageEntiers}{
\begin{question}{entiers 06} % genumsi OSUPK8
	Donner le résultat de l'addition binaire 101101 + 1011.
	\begin{multicols}{4}
	\begin{reponses}	
	\bonne{111000}
	\mauvaise{110110}
	\mauvaise{101000}
	\mauvaise{111100}
	\end{reponses}
	\end{multicols}
\end{question}}

\element{CodageEntiers}{
\begin{question}{entiers 07} % genumsi OSUPK8
	Donner l'écriture décimale du nombre binaire 10011.
	\begin{multicols}{4}
	\begin{reponses}	
	\bonne{19}
	\mauvaise{17}
	\mauvaise{23}
	\mauvaise{21}
	\end{reponses}
	\end{multicols}
\end{question}}

\element{CodageEntiers}{
\begin{question}{entiers 08} % genumsi OSUPK8
	Donner l'écriture décimale du nombre binaire 110101.
	\begin{multicols}{4}
	\begin{reponses}	
	\bonne{53}
	\mauvaise{13}
	\mauvaise{47}
	\mauvaise{51}
	\end{reponses}
	\end{multicols}
\end{question}}

\element{CodageEntiers}{
\begin{question}{entiers 09} % genumsi OSUPK8
	Donner l'écriture binaire du nombre 137.
	\begin{multicols}{4}
	\begin{reponses}	
	\bonne{10001001}
	\mauvaise{10111001}
	\mauvaise{10001010}
	\mauvaise{10010001}
	\end{reponses}
	\end{multicols}
\end{question}}

\element{CodageEntiers}{
\begin{question}{entiers 10} % genumsi OSUPK8
	Donner l'écriture binaire du nombre 34.
	\begin{multicols}{4}		
	\begin{reponses}	
	\bonne{100010}
	\mauvaise{010010}
	\mauvaise{100001}
	\mauvaise{100110}
	\end{reponses}
	\end{multicols}
\end{question}}

\element{CodageEntiers}{
\begin{question}{entiers 11} % genumsi OSUPK8
	Combien de chiffres binaires sont nécessaires pour coder le nombre 287 ?
	\begin{multicols}{4}
	\begin{reponses}	
	\bonne{9}
	\mauvaise{7}
	\mauvaise{8}
	\mauvaise{10}
	\end{reponses}
	\end{multicols}
\end{question}}

\element{CodageEntiers}{
\begin{question}{entiers 12} % genumsi OSUPK8
	Combien de chiffres possède l'écriture binaire du nombre 75 ?
	\begin{multicols}{4}
	\begin{reponses}	
	\bonne{7}
	\mauvaise{5}
	\mauvaise{6}
	\mauvaise{8}
	\end{reponses}
	\end{multicols}
\end{question}}

\element{CodageEntiers}{
\begin{question}{entiers 13} % genumsi OSUPK8
	1 octet représente combien de bit(s) ?
	\begin{multicols}{4}
	\begin{reponses}	
	\bonne{8}
	\mauvaise{2}
	\mauvaise{3}
	\mauvaise{6}
	\end{reponses}
	\end{multicols}
\end{question}}

\element{CodageEntiers}{
\begin{question}{entiers 14} % genumsi OSUPK8
	Combien faut-il de bits minimum pour représenter le nombre décimal 16 ?
	\begin{multicols}{4}
	\begin{reponses}	
	\bonne{5}
	\mauvaise{3}
	\mauvaise{4}
	\mauvaise{6}
	\end{reponses}
	\end{multicols}
\end{question}}

\element{CodageEntiers}{
\begin{question}{entiers 15} % genumsi OSUPK8
	Quelle est la valeur décimale de l'entier binaire 00011010 ? 
	\begin{multicols}{4}
	\begin{reponses}	
	\bonne{26}
	\mauvaise{32}
	\mauvaise{41}
	\mauvaise{24}
	\end{reponses}
	\end{multicols}
\end{question}}

\element{CodageEntiers}{
\begin{question}{entiers 16} % genumsi OSUPK8
	Avec 5 bits, on peut compter de .... à .... ?
	\begin{multicols}{4}
	\begin{reponses}	
	\bonne{0 à 31}
	\mauvaise{1 à 32}
	\mauvaise{0 à 32}
	\mauvaise{1 à 31}
	\end{reponses}
	\end{multicols}
\end{question}
}