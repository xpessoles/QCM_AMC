\element{StructuresFor}{
\begin{question}{for01}  % genumsi nreveret
On souhaite écrire un programme affichant tous les entiers multiples de 3 entre 6 et 288 inclus.
Quel code est correct ?
	\begin{multicols}{2}	
	\begin{reponses}	
	\bonne{$\;$\lstinputlisting{For_01_d.py}}
	\mauvaise{$\;$\lstinputlisting{For_01_a.py}}
	\mauvaise{$\;$\lstinputlisting{For_01_b.py}}
	\mauvaise{$\;$\lstinputlisting{For_01_c.py}}
	\end{reponses}
	\end{multicols}
\end{question}


\begin{question}{for02}  % genumsi nreveret
	On a saisi le code suivant :
	\lstinputlisting{For_02.py}
	Quelle est la valeur de \lstinline{a} après l’exécution du code ?
\begin{multicols}{4}	
	\begin{reponses}	
	\bonne{26}
	\mauvaise{18}
	\mauvaise{18.0}
	\mauvaise{26.0}
	\end{reponses}
	\end{multicols}
\end{question}

\begin{question}{for03}% osupk8
\texttt{Pour i allant de 0 à 9} s'écrit :	
\begin{multicols}{4}
	\begin{reponses}	
	\bonne{\lstinline{for i in range(10)}}
	\mauvaise{\lstinline{for i in range(8)}}
	\mauvaise{\lstinline{for i in range(9)}}
	\mauvaise{\lstinline{for i in range(11)}}
	\end{reponses}
	\end{multicols}
\end{question}

\begin{question}{for04}% osupk8
\texttt{pour i allant de 1 à 10} s'écrit :
\begin{multicols}{4}
	\begin{reponses}	
	\bonne{\lstinline{for i in range(1,11)}}
	\mauvaise{\lstinline{for i in range(10)}}
	\mauvaise{\lstinline{for i in range(1,10)}}
	\mauvaise{\lstinline{for i in range(0,10)}}
	\end{reponses}
	\end{multicols}
\end{question}


\begin{question}{for}% osupk8
\begin{multicols}{4}
	\begin{reponses}	
	\bonne{}
	\mauvaise{}
	\mauvaise{}
	\mauvaise{}
	\end{reponses}
	\end{multicols}
\end{question}
}