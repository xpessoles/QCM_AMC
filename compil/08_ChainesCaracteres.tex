\element{ChainesCaracteres}{
\begin{question}{str01} % genumsi nreveret
	On a saisi le code suivant : \lstinline{mot = 'première'}. Quelle affirmation est vraie ?
\begin{multicols}{4}	
	\begin{reponses}	
	\bonne{\lstinline{mot[7]} vaut \lstinline{'e'}}
	\mauvaise{\lstinline{mot[1]} vaut \lstinline{'p'}}
	\mauvaise{\lstinline{len(mot)} vaut 7}
	\mauvaise{\lstinline{len(mot)} vaut 6}
	\end{reponses}
	\end{multicols}
\end{question}

\begin{question}{str02}  % genumsi nreveret
	Quelle est le résultat de :  \lstinline{'orange'[-3] ?}
\begin{multicols}{4}	
	\begin{reponses}
	\bonne{\lstinline{'n'}}
	\mauvaise{\lstinline{'e'}}
	\mauvaise{\lstinline{'g'}}
	\mauvaise{\lstinline{Error : Negative index}}
	\end{reponses}
	\end{multicols}
\end{question}


\begin{question}{str03}  % https://www.codingame.com/
Soit le texte suivant : \lstinline{texte = "Un chasseur sachant chasser doit savoir chasser sans son chien."}. Qu'affiche la ligne suivante : 
\lstinline{print(texte[5])} ?
\begin{multicols}{4}	
	\begin{reponses}	
	\bonne{\lstinline{"a"}}
	\mauvaise{\lstinline{"h"}}
	\mauvaise{\lstinline{"s"}}
	\mauvaise{\lstinline{"Un ch"}}
	\end{reponses}
	\end{multicols}
\end{question}

\begin{question}{str04}  % https://www.codingame.com/
Soit le texte suivant : \lstinline{texte = "Un chasseur sachant chasser doit savoir chasser sans son chien."}. Qu'affiche la ligne suivante : 
\lstinline{print(texte[:5])} ?
%\begin{multicols}{4}	
	\begin{reponses}	
	\bonne{\lstinline{"Un ch"}}
	\mauvaise{\lstinline{"Un cha"}}
	\mauvaise{\lstinline{"Un chasseur sachant chasser doit"}}
	\mauvaise{\lstinline{"Un chasseur sachant chasser doit savoir chasser sans son chien."}}
	\end{reponses}
%	\end{multicols}
\end{question}

\begin{question}{str05}  % https://www.codingame.com/
Soit le texte suivant : \lstinline{texte = "Un chasseur sachant chasser doit savoir chasser sans son chien."}. Qu'affiche la ligne suivante : 
\lstinline{print(texte[5:10])} ?
\begin{multicols}{4}	
	\begin{reponses}	
	\bonne{\lstinline{"asseu"}}
	\mauvaise{\lstinline{"hasseu"}}
	\mauvaise{\lstinline{"asseur"}}
	\mauvaise{\lstinline{"ar"}}
	\end{reponses}
	\end{multicols}
\end{question}

\begin{question}{str06}  % https://www.codingame.com/
Soit le texte suivant : \lstinline{texte = "Un chasseur sachant chasser doit savoir chasser sans son chien."}. Qu'affiche la ligne suivante : 
\lstinline{print(texte[:2]+texte[6:8])} ?
\begin{multicols}{4}	
	\begin{reponses}	
	\bonne{\lstinline{"Unss"}}
	\mauvaise{\lstinline{"Un ss"}}
	\mauvaise{\lstinline{"Unsse"}}
	\mauvaise{\lstinline{"Un sse"}}
	\end{reponses}
	\end{multicols}
\end{question}

\begin{question}{str07}  % https://www.codingame.com/
Soit le texte suivant : \lstinline{texte = "Un chasseur sachant chasser doit savoir chasser sans son chien."}. 

Que faut-il mettre à la place des ... pour afficher "sachant"?
%\begin{multicols}{4}	
	\begin{reponses}	
	\bonne{\lstinline{print(texte[12:19]}}
	\mauvaise{\lstinline{print(texte[12:18]}}
	\mauvaise{\lstinline{print(texte[11:18]}}
	\mauvaise{\lstinline{print(texte[12]+texte[18])}}
	\end{reponses}
%\end{multicols}
\end{question}

\begin{question}{str08}  % genumsi planchet.d
Suite au programme ci-dessous, il faut afficher le message suivant : \texttt{je m'appelle prenom et j'ai age ans}.
\lstinputlisting{str_08.py}
Quelle instruction doit-on choisir ?

	\begin{reponses}
	\bonne{\lstinline{print("je m'appelle "+prenom+" et j'ai"+str(age)+" ans")}} 
	\mauvaise{\lstinline{print("je m'appelle "+int(prenom)+" et j'ai "+int(age)+ "ans")}}
	\mauvaise{\lstinline{print("je m'appelle "+prenom+" et j'ai "+ age+ "ans")}}
 	\mauvaise{\lstinline{print("je m'appelle "+prenom+" et j'ai "+int(age)+ "ans")}}
	\end{reponses}
\end{question}

\begin{question}{str}  % genumsi nreveret

\begin{multicols}{4}	
	\begin{reponses}	
	\bonne{}
	\mauvaise{}
	\mauvaise{}
	\mauvaise{}
	\end{reponses}
	\end{multicols}
\end{question}
}

