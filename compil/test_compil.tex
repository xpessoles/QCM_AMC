\documentclass[a4paper]{article}

% bloc : évite un changement de page
% completemulti : ajoute une case : aucune bonne réponse
%\newcommand{\repRel}{../..}
%\input{\repRel/Style/packages}
%%\input{\repRel/Style/new_style}
%\input{\repRel/Style/macros_SII}
%%\input{\repRel/Style/environment}

%\usepackage[francais,bloc,completemulti,ensemble]{automultiplechoice}
\usepackage[francais,bloc,completemulti,catalog]{automultiplechoice} %ensemble
\usepackage{multicol}

%% Pour le python %%
\usepackage{listingsutf8}

\lstset{language=Python,
  inputencoding=utf8/latin1,
  breaklines=true,
  basicstyle=\ttfamily\small,
  keywordstyle=\bfseries\color{green!40!black},
  commentstyle=\itshape\color{purple!40!black},
  identifierstyle=\color{blue},
  stringstyle=\color{orange},
  upquote = true,
  columns=fullflexible,
  backgroundcolor=\color{gray!10},frame=leftline,rulecolor=\color{gray}}  
  
\definecolor{mygreen}{rgb}{0,0.6,0}

\lstset{
     literate=%
         {é}{{\'e}}1    
         {è}{{\`e}}1    
         {ê}{{\^e}}1    
         {à}{{\`a}}1
         {â}{{\^a}}1		 
         {ô}{{\^o}}1    
         {ù}{{\`u}}1    
         {î}{{\^i}}1    
}
\lstset{inputpath=code}

%\usepackage[utf8x]{inputenc}
%\usepackage[T1]{fontenc}

\begin{document}


\element{OperationsElementaires}{
\begin{question}{OpEl01}% genumsi nreveret
	On souhaite écrire un programme calculant le triple d'un nombre décimal et affichant le résultat. On a saisi le code suivant :
	\lstinputlisting{OpEl_01.py}
	Quel va être le résultat affiché ?
\begin{multicols}{4}
	\begin{reponses}	
	\bonne{555}
	\mauvaise{\lstinline{nombrenombrenombre}}
	\mauvaise{\lstinline{15}}
	\mauvaise{\lstinline{15.0}}
	\end{reponses}
	\end{multicols}
\end{question}

\begin{question}{OpEl02} % nreveret
	On a saisi le code suivant : 
	\lstinputlisting{OpEl_02.py}
	Quelle instruction permet d’afficher le message \lstinline{1 + 1 = 2} ?
\begin{multicols}{4}	
	\begin{reponses}	
	\bonne{\lstinline{print(a + ' = ' + c)}}
	\mauvaise{\lstinline{print(a + ' = ' + d)}}
	\mauvaise{\lstinline{print(b + ' = ' + c)}}
	\mauvaise{\lstinline{print(b + ' = ' + d)}}
	\end{reponses}
	\end{multicols}
\end{question}

\begin{question}{OpEl03} % nreveret
	On a saisi le code suivant : 
	\lstinputlisting{OpEl_03.py}
	Quelles sont les valeurs de a et b à la fin du programme ?
\begin{multicols}{4}	
	\begin{reponses}	
	\bonne{\lstinline{a = 5} et \lstinline{b = 8}}
	\mauvaise{\lstinline{a = 8} et \lstinline{b = 5}}
	\mauvaise{\lstinline{a = 8} et \lstinline{b = 13}}
	\mauvaise{\lstinline{a = 13} et \lstinline{b = 5}}
	\end{reponses}
	\end{multicols}
\end{question}


\begin{question}{OpEl}%

\begin{multicols}{4}	
	\begin{reponses}	
	\bonne{}
	\mauvaise{}
	\mauvaise{}
	\mauvaise{}
	\end{reponses}
	\end{multicols}
\end{question}
}


\element{OperationsArithmetiques}{
\begin{question}{OpAr01}
	On exécute l'instruction ci-après. Quel est l'affichage attendu ?
	\lstinputlisting{oparith01.py}
	\begin{multicols}{4}
	\begin{reponses}	
	\bonne{0}
	\mauvaise{1}
	\mauvaise{2}
	\mauvaise{4}
	\end{reponses}
	\end{multicols}
\end{question}}


\element{CodageEntiers}{
\begin{question}{entiers 01} % genumsi nreveret
	Quel est l'entier positif codé en base 2 sur 8 bits par le code 0011 1010 ?
	\begin{multicols}{4}
	\begin{reponses}	
	\bonne{58}
	\mauvaise{45}
	\mauvaise{25}
	\mauvaise{-12}
	\end{reponses}
	\end{multicols}
\end{question}

\begin{question}{entiers 02} % genumsi nreveret
	Le résultat de l'addition des deux nombres binaires 1101 et 0101 est:
	\begin{multicols}{4}
	\begin{reponses}	
	\bonne{10010}
	\mauvaise{10110}
	\mauvaise{10011}
	\mauvaise{11010}
	\end{reponses}
	\end{multicols}
\end{question}

\begin{question}{entiers 03} % genumsi nreveret
	Convertir  la valeur décimale 155 en binaire (sur un octet).
	\begin{multicols}{4}
	\begin{reponses}	
	\bonne{10011011}
	\mauvaise{11011011}
	\mauvaise{01111111}
	\mauvaise{10010111}
	\end{reponses}
	\end{multicols}
\end{question}
}

\element{CodageEntiersRelatifs}{
\begin{question}{relatifs 01}% genumsi nreveret
	Quel est l'entier relatif codé en complément à 2 sur un octet par le code 1111 1111 ?
	\begin{multicols}{4}
	\begin{reponses}	
	\bonne{-1}
	\mauvaise{255}
	\mauvaise{127}
	\mauvaise{45}
	\end{reponses}
	\end{multicols}
\end{question}}


\element{CodageHexa}{
\begin{question}{hexa 01} % genumsi nreveret
	Convertir  la valeur décimale 195 en hexadécimal.
	\begin{multicols}{4}
	\begin{reponses}
	\bonne{C3}
	\mauvaise{A5}
	\mauvaise{B9}
	\mauvaise{C9}
	\end{reponses}
	\end{multicols}
\end{question}

\begin{question}{hexa 01}

	\begin{multicols}{4}
	\begin{reponses}	
	\bonne{}
	\mauvaise{}
	\mauvaise{}
	\mauvaise{}
	\end{reponses}
	\end{multicols}
\end{question}
}

\element{AlgebreBoole}{
\begin{question}{boole 01}% genumsi nreveret
	En logique (algèbre de Boole), l'expression: \lstinline{not (A or B)} est équivalente à  :
	\begin{multicols}{4}
	\begin{reponses}	
	\bonne{\lstinline{(not A) and (not B)}}
	\mauvaise{\lstinline{(not A) or (not B)}}
	\mauvaise{\lstinline{A or B}}
	\mauvaise{\lstinline{A and B}}
	\end{reponses}
	\end{multicols}
\end{question}}


\element{ChainesCaracteres}{
\begin{question}{str01} % genumsi nreveret
	On a saisi le code suivant :\lstinline{mot = 'première'}. Quelle affirmation est vraie ?
\begin{multicols}{4}	
	\begin{reponses}	
	\bonne{\lstinline{mot[7]} vaut \lstinline{'e'}}
	\mauvaise{\lstinline{mot[1]} vaut \lstinline{'p'}}
	\mauvaise{\lstinline{len(mot)} vaut 7}
	\mauvaise{\lstinline{len(mot)} vaut 6}
	\end{reponses}
	\end{multicols}
\end{question}


\begin{question}{str}

\begin{multicols}{4}	
	\begin{reponses}	
	\bonne{}
	\mauvaise{}
	\mauvaise{}
	\mauvaise{}
	\end{reponses}
	\end{multicols}
\end{question}
}



\element{StructuresConditionnelles}{
\begin{question}{}

\begin{multicols}{4}	
	\begin{reponses}	
	\bonne{}
	\mauvaise{}
	\mauvaise{}
	\mauvaise{}
	\end{reponses}
	\end{multicols}
\end{question}
}



\element{StructuresTantQue}{
\begin{question}{TtQue01}  % genumsi nreveret
On a saisi le code suivant :
	\lstinputlisting{TtQue_01.py}
	Quelle est la valeur de \lstinline{n} après l’exécution du code ?
\begin{multicols}{4}	
	\begin{reponses}	
	\bonne{1.0}
	\mauvaise{4.0}
	\mauvaise{2.0}
	\mauvaise{0.5}
	\end{reponses}
	\end{multicols}
\end{question}
\begin{question}{}

\begin{multicols}{4}	
	\begin{reponses}	
	\bonne{}
	\mauvaise{}
	\mauvaise{}
	\mauvaise{}
	\end{reponses}
	\end{multicols}
\end{question}
}

\element{StructuresFor}{
\begin{question}{for01}  % genumsi nreveret
On souhaite écrire un programme affichant tous les entiers multiples de 3 entre 6 et 288 inclus.
Quel code est correct ?
	\begin{multicols}{2}	
	\begin{reponses}	
	\bonne{\lstinputlisting{For_01_d.py}}
	\mauvaise{\lstinputlisting{For_01_a.py}}
	\mauvaise{\lstinputlisting{For_01_b.py}}
	\mauvaise{\lstinputlisting{For_01_c.py}}
	\end{reponses}
	\end{multicols}
\end{question}


\begin{question}{for02}  % genumsi nreveret
	On a saisi le code suivant :
	\lstinputlisting{For_02.py}
	Quelle est la valeur de \lstinline{a} après l’exécution du code ?
\begin{multicols}{4}	
	\begin{reponses}	
	\bonne{26}
	\mauvaise{18}
	\mauvaise{18.0}
	\mauvaise{26.0}
	\end{reponses}
	\end{multicols}
\end{question}

\begin{question}{for}

\begin{multicols}{4}	
	\begin{reponses}	
	\bonne{}
	\mauvaise{}
	\mauvaise{}
	\mauvaise{}
	\end{reponses}
	\end{multicols}
\end{question}
}


\element{listes}{
\begin{question}{list01}  % genumsi nreveret
Voici une fonction Python de recherche d'un maximum :

Avec quelle précondition sur la liste \lstinline{t}, la postcondition « <i>m est un élément maximum de la liste s</i> » n'est-elle pas assurée ?</p>
\begin{multicols}{4}	
	\begin{reponses}	
	\bonne{}
	\mauvaise{}
	\mauvaise{}
	\mauvaise{}
	\end{reponses}
	\end{multicols}
\end{question}
}

\element{FonctionsEtListes}{
\begin{question}{fonclist01}  % genumsi nreveret
Voici une fonction Python de recherche d'un maximum :
\lstinputlisting{FoncList_01.py}
Avec quelle précondition sur la liste \lstinline{t}, la postcondition `` \lstinline{m} est un élément maximum de la liste \lstinline{s}'' n'est-elle pas assurée ?
	\begin{reponses}
	\bonne{Tout élément de \lstinline{t} est un entier supérieur ou égal à -2.}
	\mauvaise{Tout élément de \lstinline{t} est un entier positif ou nul.}
	\mauvaise{Tout élément de \lstinline{t} est un entier supérieur ou égal à 11.}
	\mauvaise{Tout élément de \lstinline{t} est un entier strictement supérieur à -2.}
	\end{reponses}
\end{question}
}



\element{}{
\begin{question}{}

\begin{multicols}{4}	
	\begin{reponses}	
	\bonne{}
	\mauvaise{}
	\mauvaise{}
	\mauvaise{}
	\end{reponses}
	\end{multicols}
\end{question}
}




%
%\begin{question}{bin_}
%	\begin{multicols}{4}
%	\begin{reponses}	
%	\bonne{}
%	\mauvaise{}
%	\mauvaise{}
%	\mauvaise{}
%	\end{reponses}
%	\end{multicols}
%\end{question}
%}


	\melangegroupe{OperationsElementaires}
	\restituegroupe{OperationsElementaires}

	\melangegroupe{OperationsArithmetiques}
	\restituegroupe{OperationsArithmetiques}

	\melangegroupe{CodageEntiers}
	\restituegroupe{CodageEntiers}
	
	\melangegroupe{CodageEntiersRelatifs}
	\restituegroupe{CodageEntiersRelatifs}

	\melangegroupe{CodageHexa}
	\restituegroupe{CodageHexa}
	
	\melangegroupe{AlgebreBoole}
	\restituegroupe{AlgebreBoole}

	\melangegroupe{ChainesCaracteres}
	\restituegroupe{ChainesCaracteres}
	
	\melangegroupe{StructuresConditionnelles}
	\restituegroupe{StructuresConditionnelles}
	
	\melangegroupe{StructuresTantQue}
	\restituegroupe{StructuresTantQue}
	
	\melangegroupe{StructuresFor}
	\restituegroupe{StructuresFor}
	
	\melangegroupe{FonctionsEtListes}
	\restituegroupe{FonctionsEtListes}
\end{document}