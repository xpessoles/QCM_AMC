\documentclass[a4paper]{article}

% bloc : évite un changement de page
% completemulti : ajoute une case : aucune bonne réponse
%\newcommand{\repRel}{../..}
%\input{\repRel/Style/packages}
%%\input{\repRel/Style/new_style}
%\input{\repRel/Style/macros_SII}
%%\input{\repRel/Style/environment}

\usepackage[francais,bloc,completemulti,ensemble]{automultiplechoice}

%% Pour le python %%
\usepackage{listingsutf8}

\lstset{language=Python,
  inputencoding=utf8/latin1,
  breaklines=true,
  basicstyle=\ttfamily\small,
  keywordstyle=\bfseries\color{green!40!black},
  commentstyle=\itshape\color{purple!40!black},
  identifierstyle=\color{blue},
  stringstyle=\color{orange},
  upquote = true,
  columns=fullflexible,
  backgroundcolor=\color{gray!10},frame=leftline,rulecolor=\color{gray}}  
  
\definecolor{mygreen}{rgb}{0,0.6,0}

\lstset{
     literate=%
         {é}{{\'e}}1    
         {è}{{\`e}}1    
         {ê}{{\^e}}1    
         {à}{{\`a}}1
         {â}{{\^a}}1		 
         {ô}{{\^o}}1    
         {ù}{{\`u}}1    
         {î}{{\^i}}1    
}


\usepackage[utf8x]{inputenc}
\usepackage[T1]{fontenc}

\begin{document}


\element{opArith}{
\begin{question}{opArith01Q01}
	On exécute l'instruction ci-après. Quel est l'affichage attendu ?
	\begin{lstlisting}
	>>> 4%2
	\end{lstlisting}
	\begin{reponseshoriz}
	\bonne{0}
	\mauvaise{1}
	\mauvaise{2}
	\mauvaise{4}
\end{reponseshoriz}
\end{question}}

\element{opArith}{
\begin{question}{opArith01Q02}
	On exécute l'instruction ci-après. Quel est l'affichage attendu ?
	\begin{lstlisting}
	>>> 4%2
	\end{lstlisting}
	\begin{reponseshoriz}
	\bonne{1} 
	\mauvaise{0}
	\mauvaise{2}
	\mauvaise{4}
\end{reponseshoriz}
\end{question}}
%}

\restituegroupe{opArith}

TEST

\end{document}