\documentclass[a4paper]{article}

% bloc : évite un changement de page
% completemulti : ajoute une case : aucune bonne réponse
%\newcommand{\repRel}{../..}
%\input{\repRel/Style/packages}
%%\input{\repRel/Style/new_style}
%\input{\repRel/Style/macros_SII}
%%\input{\repRel/Style/environment}

%\usepackage[francais,bloc,completemulti,ensemble]{automultiplechoice}
\usepackage[francais,bloc,completemulti,catalog]{automultiplechoice} %ensemble
\usepackage{multicol}

%% Pour le python %%
\usepackage{listingsutf8}

\lstset{language=Python,
  inputencoding=utf8/latin1,
  breaklines=true,
  basicstyle=\ttfamily\small,
  keywordstyle=\bfseries\color{green!40!black},
  commentstyle=\itshape\color{purple!40!black},
  identifierstyle=\color{blue},
  stringstyle=\color{orange},
  upquote = true,
  columns=fullflexible,
  backgroundcolor=\color{gray!10},frame=leftline,rulecolor=\color{gray}}  
  
\definecolor{mygreen}{rgb}{0,0.6,0}

\lstset{
     literate=%
         {é}{{\'e}}1    
         {è}{{\`e}}1    
         {ê}{{\^e}}1    
         {à}{{\`a}}1
         {â}{{\^a}}1		 
         {ô}{{\^o}}1    
         {ù}{{\`u}}1    
         {î}{{\^i}}1    
}
\lstset{inputpath=code}

%\usepackage[utf8x]{inputenc}
%\usepackage[T1]{fontenc}

\begin{document}

\element{OperationsElementaires}{
\begin{question}{OpEl 01}% genumsi nreveret
	On souhaite écrire un programme calculant le triple d'un nombre décimal et affichant le résultat. On a saisi le code suivant :
	\lstinputlisting{OpEl_01.py}
	Quel va être le résultat affiché ?
	\begin{reponses}	
	\bonne{555}
	\mauvaise{\lstinline{nombrenombrenombre}}
	\mauvaise{\lstinline{15}}
	\mauvaise{\lstinline{15.0}}
	\end{reponses}
\end{question}\\}

\element{OperationsElementaires}{
\begin{question}{OpEl 02} % nreveret
	On a saisi le code suivant : 
	\lstinputlisting{OpEl_02.py}
	Quelle instruction permet d’afficher le message \lstinline{1 + 1 = 2} ?
\begin{multicols}{2}	
	\begin{reponses}	
	\bonne{\lstinline{print(a + ' = ' + c)}}
	\mauvaise{\lstinline{print(a + ' = ' + d)}}
	\mauvaise{\lstinline{print(b + ' = ' + c)}}
	\mauvaise{\lstinline{print(b + ' = ' + d)}}
	\end{reponses}
	\end{multicols}
\end{question}\\}

\element{OperationsElementaires}{
\begin{question}{OpEl 03} % nreveret
	On a saisi le code suivant : 
	\lstinputlisting{OpEl_03.py}
	Quelles sont les valeurs de a et b à la fin du programme ?
\begin{multicols}{4}	
	\begin{reponses}	
	\bonne{\lstinline{a = 5} et \lstinline{b = 8}}
	\mauvaise{\lstinline{a = 8} et \lstinline{b = 5}}
	\mauvaise{\lstinline{a = 8} et \lstinline{b = 13}}
	\mauvaise{\lstinline{a = 13} et \lstinline{b = 5}}
	\end{reponses}
	\end{multicols}
\end{question}\\}

\element{OperationsElementaires}{
\begin{question}{OpEl 04} % osupk8
Que contient la variable \lstinline{a} si on execute ce script ?
\lstinputlisting{opel_04.py}
\begin{multicols}{4}	
	\begin{reponses}
	\bonne{-5.0}
	\mauvaise{5.0}
	\mauvaise{1.0}
	\mauvaise{-1.0}
	\end{reponses}
	\end{multicols}
\end{question}\\}

\element{OperationsElementaires}{
\begin{question}{OpEl 05} % osupk8
Que contient la variable \lstinline{a} si on exécute ce script ?
	\lstinputlisting{OpEl_05.py}
\begin{multicols}{4}	
	\begin{reponses}	
	\bonne{16.0}
	\mauvaise{9.0}
	\mauvaise{10.0}
	\mauvaise{12.0}
	\end{reponses}
	\end{multicols}
\end{question}\\}

\element{OperationsElementaires}{
\begin{question}{OpEl 06} % osupk8
Que contiennent les variables \lstinline{a} et \lstinline{b} si on execute ce script ?
	\lstinputlisting{OpEl_06.py}
\begin{multicols}{4}	
	\begin{reponses}
	\bonne{5.0 et 7.0}
	\mauvaise{5.0 et 5.0}
	\mauvaise{7.0 et 5.0}
	\mauvaise{7.0 et 7.0}
	\end{reponses}
	\end{multicols}
\end{question}\\}

\element{OperationsElementaires}{
\begin{question}{OpEl 07}%% osupk8
Que taper en Python pour obtenir $3^8$ ?
\begin{multicols}{4}	
	\begin{reponses}	
	\bonne{\lstinline{3**8}}
	\mauvaise{\lstinline{3^8}}
	\mauvaise{\lstinline{3*8}}
	\mauvaise{\lstinline{3&8}}
	\end{reponses}
	\end{multicols}
\end{question}\\}

\element{OperationsElementaires}{
\begin{question}{OpEl 08}%% planchet.d
On a saisi le code suivant : \lstinline{a = '8'} puis \lstinline{b = 5} et \lstinline{a + b}. Que retourne ce programme~?
	\begin{reponses}	
	\bonne{TypeError : must be str, not int.}
	\mauvaise{'13'}
	\mauvaise{False}
	\mauvaise{13}
	\end{reponses}
\end{question}\\}

\element{OperationsElementaires}{
\begin{question}{OpEl 09}%% planchet.d
On souhaite écrire un programme calculant le triple d'un nombre décimal et affichant le résultat. On a saisi le code suivant : \lstinline{nombre = '5'} puis \lstinline{triple = nombre * 3}. Quel va être le résultat affiché en saisissant \lstinline{print(triple)}?

	\begin{reponses}	
	\bonne{555}
	\mauvaise{15}
	\mauvaise{15.0}
	\mauvaise{nombrenombrenombre}
	\end{reponses}

\end{question}\\}

\element{OperationsElementaires}{
\begin{question}{OpEl 10}%% sgenre
En python, que fait l'instruction suivante ? \lstinline{#print(a,b)}
	\begin{reponses}	
	\bonne{Elle ne fait rien.}
	\mauvaise{Elle affiche le texte 'a,b'.} 
	\mauvaise{Elle affiche les valeurs de a et b.}
	\mauvaise{Elle génère une erreur.}
	\end{reponses}
\end{question}\\}

\element{OperationsElementaires}{
\begin{question}{OpEl 11}% sgenre
En python, combien vaut : \texttt{12\%5} ?
	\begin{reponses}	
	\bonne{3}
	\mauvaise{1}
	\mauvaise{3}
	\mauvaise{Ce calcul génère une erreur de calcul.}
	\end{reponses}
\end{question}\\}
%
%\element{OperationsElementaires}{
%\begin{question}{OpEl}% sgenre
%
%\begin{multicols}{4}	
%	\begin{reponses}	
%	\bonne{}
%	\mauvaise{}
%	\mauvaise{}
%	\mauvaise{}
%	\end{reponses}
%	\end{multicols}
%\end{question}
%\\}
\element{OperationsArithmetiques}{
\begin{question}{OpAr01}
	On exécute l'instruction ci-après. Quel est l'affichage attendu ?
	\lstinputlisting{oparith01.py}
	\begin{multicols}{4}
	\begin{reponses}	
	\bonne{0}
	\mauvaise{1}
	\mauvaise{2}
	\mauvaise{4}
	\end{reponses}
	\end{multicols}
\end{question}
\begin{question}{OpAr}
	
	%\lstinputlisting{oparith01.py}
	\begin{multicols}{4}
	\begin{reponses}	
	\bonne{}
	\mauvaise{}
	\mauvaise{}
	\mauvaise{}
	\end{reponses}
	\end{multicols}
\end{question}}

\element{Fonctions}{
\begin{question}{fon01} % osupk8
Avec la fonction donnée ci-dessous l'instruction \lstinline{mystere(0,1)} retourne : 
\lstinputlisting{fon_01.py}
\begin{multicols}{4}	
	\begin{reponses}	
	\bonne{1}
	\mauvaise{0}
	\mauvaise{True}
	\mauvaise{False}
	\end{reponses}
	\end{multicols}
\end{question}

\begin{question}{fon}

\begin{multicols}{4}	
	\begin{reponses}	
	\bonne{}
	\mauvaise{}
	\mauvaise{}
	\mauvaise{}
	\end{reponses}
	\end{multicols}
\end{question}
}

\element{CodageEntiers}{
\begin{question}{entiers 01} % genumsi nreveret
	Quel est l'entier positif codé en base 2 sur 8 bits par le code 0011 1010 ?
	\begin{multicols}{4}
	\begin{reponses}	
	\bonne{58}
	\mauvaise{45}
	\mauvaise{25}
	\mauvaise{-12}
	\end{reponses}
	\end{multicols}
\end{question}

\begin{question}{entiers 02} % genumsi nreveret
	Le résultat de l'addition des deux nombres binaires 1101 et 0101 est:
	\begin{multicols}{4}
	\begin{reponses}	
	\bonne{10010}
	\mauvaise{10110}
	\mauvaise{10011}
	\mauvaise{11010}
	\end{reponses}
	\end{multicols}
\end{question}

\begin{question}{entiers 03} % genumsi nreveret
	Convertir  la valeur décimale 155 en binaire (sur un octet).
	\begin{multicols}{4}
	\begin{reponses}	
	\bonne{10011011}
	\mauvaise{11011011}
	\mauvaise{01111111}
	\mauvaise{10010111}
	\end{reponses}
	\end{multicols}
\end{question}


\begin{question}{entiers 04} % genumsi OSUPK8
	Quelle est la valeur décimale de l'entier binaire 00011010 ?
	\begin{multicols}{4}
	\begin{reponses}	
	\bonne{26}
	\mauvaise{22}
	\mauvaise{51}
	\mauvaise{24}
	\end{reponses}
	\end{multicols}
\end{question}

\begin{question}{entiers 05} % genumsi OSUPK8
	Donner le résultat de l'addition binaire : 1101 + 1001.

	\begin{multicols}{4}
	\begin{reponses}	
	\bonne{10110}
	\mauvaise{01001}
	\mauvaise{00110}
	\mauvaise{11010}
	\end{reponses}
	\end{multicols}
\end{question}

\begin{question}{entiers 06} % genumsi OSUPK8
	Donner le résultat de l'addition binaire 101101 + 1011.
	\begin{multicols}{4}
	\begin{reponses}	
	\bonne{111000}
	\mauvaise{110110}
	\mauvaise{101000}
	\mauvaise{111100}
	\end{reponses}
	\end{multicols}
\end{question}

\begin{question}{entiers 07} % genumsi OSUPK8
	Donner l'écriture décimale du nombre binaire 10011.
	\begin{multicols}{4}
	\begin{reponses}	
	\bonne{19}
	\mauvaise{17}
	\mauvaise{23}
	\mauvaise{21}
	\end{reponses}
	\end{multicols}
\end{question}

\begin{question}{entiers 08} % genumsi OSUPK8
	Donner l'écriture décimale du nombre binaire 110101.
	\begin{multicols}{4}
	\begin{reponses}	
	\bonne{53}
	\mauvaise{13}
	\mauvaise{47}
	\mauvaise{51}
	\end{reponses}
	\end{multicols}
\end{question}

\begin{question}{entiers 09} % genumsi OSUPK8
	Donner l'écriture binaire du nombre 137.
	\begin{multicols}{4}
	\begin{reponses}	
	\bonne{10001001}
	\mauvaise{10111001}
	\mauvaise{10001010}
	\mauvaise{10010001}
	\end{reponses}
	\end{multicols}
\end{question}

\begin{question}{entiers 10} % genumsi OSUPK8
	Donner l'écriture binaire du nombre 34.
	\begin{multicols}{4}		
	\begin{reponses}	
	\bonne{100010}
	\mauvaise{010010}
	\mauvaise{100001}
	\mauvaise{100110}
	\end{reponses}
	\end{multicols}
\end{question}

\begin{question}{entiers 11} % genumsi OSUPK8
	Combien de chiffres binaires sont nécessaires pour coder le nombre 287 ?
	\begin{multicols}{4}
	\begin{reponses}	
	\bonne{9}
	\mauvaise{7}
	\mauvaise{8}
	\mauvaise{10}
	\end{reponses}
	\end{multicols}
\end{question}


\begin{question}{entiers 12} % genumsi OSUPK8
	Combien de chiffres possède l'écriture binaire du nombre 75 ?
	\begin{multicols}{4}
	\begin{reponses}	
	\bonne{7}
	\mauvaise{5}
	\mauvaise{6}
	\mauvaise{8}
	\end{reponses}
	\end{multicols}
\end{question}

\begin{question}{entiers 13} % genumsi OSUPK8
	1 octet représente combien de bit(s) ?
	\begin{multicols}{4}
	\begin{reponses}	
	\bonne{8}
	\mauvaise{2}
	\mauvaise{3}
	\mauvaise{6}
	\end{reponses}
	\end{multicols}
\end{question}

\begin{question}{entiers 14} % genumsi OSUPK8
	Combien faut-il de bits minimum pour représenter le nombre décimal 16 ?
	\begin{multicols}{4}
	\begin{reponses}	
	\bonne{5}
	\mauvaise{3}
	\mauvaise{4}
	\mauvaise{6}
	\end{reponses}
	\end{multicols}
\end{question}

\begin{question}{entiers 15} % genumsi OSUPK8
	Quelle est la valeur décimale de l'entier binaire 00011010 ? 
	\begin{multicols}{4}
	\begin{reponses}	
	\bonne{26}
	\mauvaise{32}
	\mauvaise{41}
	\mauvaise{24}
	\end{reponses}
	\end{multicols}
\end{question}

\begin{question}{entiers 16} % genumsi OSUPK8
	Avec 5 bits, on peut compter de .... à .... ?
	\begin{multicols}{4}
	\begin{reponses}	
	\bonne{0 à 31}
	\mauvaise{1 à 32}
	\mauvaise{0 à 32}
	\mauvaise{1 à 31}
	\end{reponses}
	\end{multicols}
\end{question}
}
\element{CodageEntiersRelatifs}{
\begin{question}{relatifs 01}% genumsi nreveret
	Quel est l'entier relatif codé en complément à 2 sur un octet par le code 1111 1111 ?
	\begin{multicols}{4}
	\begin{reponses}	
	\bonne{-1}
	\mauvaise{255}
	\mauvaise{127}
	\mauvaise{45}
	\end{reponses}
	\end{multicols}
\end{question}}

\element{CodageHexa}{
\begin{question}{hexa 01} % genumsi nreveret
	Convertir  la valeur décimale 195 en hexadécimal.
	\begin{multicols}{4}
	\begin{reponses}
	\bonne{C3}
	\mauvaise{A5}
	\mauvaise{B9}
	\mauvaise{C9}
	\end{reponses}
	\end{multicols}
\end{question}

\begin{question}{hexa 02}% genumsi OSUPK8
	Donner l'écriture hexadécimale du nombre binaire 1001011.
	\begin{multicols}{4}
	\begin{reponses}	
	\bonne{4B}
	\mauvaise{3D}
	\mauvaise{49}
	\mauvaise{5B}
	\end{reponses}
	\end{multicols}
\end{question}

\begin{question}{hexa 03}% genumsi OSUPK8
	Donner l'écriture hexadécimale du nombre binaire 110101.
	\begin{multicols}{4}
	\begin{reponses}	
	\bonne{35}
	\mauvaise{6B}
	\mauvaise{65}
	\mauvaise{56}
	\end{reponses}
	\end{multicols}
\end{question}

\begin{question}{hexa 04}% genumsi OSUPK8
	Donner l'écriture binaire du nombre hexadécimal 6E.
	\begin{multicols}{4}
	\begin{reponses}	
	\bonne{01101110}
	\mauvaise{01110110}
	\mauvaise{01101101}
	\mauvaise{01110010}
	\end{reponses}
	\end{multicols}
\end{question}

\begin{question}{hexa 05}% genumsi OSUPK8
	Donner l'écriture binaire du nombre hexadécimal B5.
	\begin{multicols}{4}
	\begin{reponses}	
	\bonne{10110101}
	\mauvaise{10110111}
	\mauvaise{00110101}
	\mauvaise{10111101}
	\end{reponses}
	\end{multicols}
\end{question}

\begin{question}{hexa 06}% genumsi OSUPK8
	Quelle est la représentation binaire du nombre $5D_{16}$ ?
	\begin{multicols}{4}
	\begin{reponses}	
	\bonne{01011101}
	\mauvaise{01101101}
	\mauvaise{10101101}
	\mauvaise{01011110}
	\end{reponses}
	\end{multicols}
\end{question}

\begin{question}{hexa 07}% genumsi OSUPK8
	Quelle est la valeur hexadécimale de l'entier binaire 10110110 ?
	\begin{multicols}{4}
	\begin{reponses}
	\bonne{B6}
	\mauvaise{C4}
	\mauvaise{B8}
	\mauvaise{C6}
	\end{reponses}
	\end{multicols}
\end{question}

\begin{question}{hexa 08}% genumsi OSUPK8
	\begin{multicols}{4}
	\begin{reponses}	
	\bonne{}
	\mauvaise{}
	\mauvaise{}
	\mauvaise{}
	\end{reponses}
	\end{multicols}
\end{question}

\begin{question}{hexa }% genumsi OSUPK8
	\begin{multicols}{4}
	\begin{reponses}	
	\bonne{}
	\mauvaise{}
	\mauvaise{}
	\mauvaise{}
	\end{reponses}
	\end{multicols}
\end{question}

\begin{question}{hexa }% genumsi OSUPK8
	\begin{multicols}{4}
	\begin{reponses}	
	\bonne{}
	\mauvaise{}
	\mauvaise{}
	\mauvaise{}
	\end{reponses}
	\end{multicols}
\end{question}
}


\element{AlgebreBoole}{
\begin{question}{boole01}% genumsi nreveret
	En logique (algèbre de Boole), l'expression: \lstinline{not (A or B)} est équivalente à  :
	\begin{multicols}{4}
	\begin{reponses}	
	\bonne{\lstinline{(not A) and (not B)}}
	\mauvaise{\lstinline{(not A) or (not B)}}
	\mauvaise{\lstinline{A or B}}
	\mauvaise{\lstinline{A and B}}
	\end{reponses}
	\end{multicols}
\end{question}

\begin{question}{boole02}% genumsi osupk8
	Laquelle de ces propriétés est toujours vraie ?
	\begin{multicols}{4}
	\begin{reponses}	
	\bonne{\lstinline{a and  (not a) == False}}
	\mauvaise{\lstinline{a and (not a) == True}}
	\mauvaise{\lstinline{a and  (not a) == not  a}}
	\mauvaise{\lstinline{a and  (not a) == a}}
	\end{reponses}
	\end{multicols}
\end{question}

\begin{question}{boole }% genumsi nreveret

	\begin{multicols}{4}
	\begin{reponses}	
	\bonne{}
	\mauvaise{}
	\mauvaise{}
	\mauvaise{}
	\end{reponses}
	\end{multicols}
\end{question}
}
\element{ChainesCaracteres}{
\begin{question}{str01} % genumsi nreveret
	On a saisi le code suivant : \lstinline{mot = 'première'}. Quelle affirmation est vraie ?
\begin{multicols}{4}	
	\begin{reponses}	
	\bonne{\lstinline{mot[7]} vaut \lstinline{'e'}}
	\mauvaise{\lstinline{mot[1]} vaut \lstinline{'p'}}
	\mauvaise{\lstinline{len(mot)} vaut 7}
	\mauvaise{\lstinline{len(mot)} vaut 6}
	\end{reponses}
	\end{multicols}
\end{question}

\begin{question}{str}  % genumsi nreveret
	Quelle est le résultat de :  \lstinline{'orange'[-3] ?}
\begin{multicols}{4}	
	\begin{reponses}
	\bonne{\lstinline{'n'}}
	\mauvaise{\lstinline{'e'}}
	\mauvaise{\lstinline{'g'}}
	\mauvaise{\lstinline{Error : Negative index}}
	\end{reponses}
	\end{multicols}
\end{question}


\begin{question}{str}  % genumsi nreveret

\begin{multicols}{4}	
	\begin{reponses}	
	\bonne{}
	\mauvaise{}
	\mauvaise{}
	\mauvaise{}
	\end{reponses}
	\end{multicols}
\end{question}
}


\element{StructuresTantQue}{
\begin{question}{TtQue01}  % genumsi nreveret
On a saisi le code suivant :
	\lstinputlisting{TtQue_01.py}
	Quelle est la valeur de \lstinline{n} après l’exécution du code ?
\begin{multicols}{4}	
	\begin{reponses}	
	\bonne{1.0}
	\mauvaise{4.0}
	\mauvaise{2.0}
	\mauvaise{0.5}
	\end{reponses}
	\end{multicols}
\end{question}\\}

\element{StructuresTantQue}{
\begin{question}{TtQue02}  % genumsi nreveret
Après le code Python qui suit, quelles sont les valeurs finales de x et de y ? 
	\lstinputlisting{TtQue_02.py}
	\begin{reponses}	
	\bonne{La valeur finale de x est 0 et celle de y est 1.}
	\mauvaise{La valeur finale de x est -1 et celle de y est 0.}
	\mauvaise{La valeur finale de x est 0 et celle de y est 0.}
	\mauvaise{La boucle externe est une boucle infinie, le programme ne termine pas.}
	\end{reponses}
\end{question}\\}

\element{StructuresTantQue}{
\begin{question}{TtQue03}  % genumsi nreveret
On a saisi le code suivant : 
	\lstinputlisting{TtQue_03.py}
	Quelle affirmation est vraie dans celles proposées ci-dessous ?
	\begin{reponses}	
	\bonne{On sort de la boucle while dès que nombre  $\leq$ 5.}
	\mauvaise{On sort de la boucle while dès que nombre $> 5$.}
	\mauvaise{On sort de la boucle while dès que nombre $< 5$.}
	\mauvaise{On continue la boucle tant que nombre  $\leq 5$.}
	\end{reponses}
\end{question}\\}

\element{StructuresTantQue}{
\begin{question}{TtQue04}
Après le code Python qui suit, quelles sont les valeurs finales de x et de y ?
	\lstinputlisting{TtQue_04.py}
	\begin{reponses}
	\bonne{La valeur finale de x est 3 et celle de y est 4.}
	\mauvaise{La valeur finale de x est 3 et celle de y est 3.}
	\mauvaise{La valeur finale de x est 4 et celle de y est 3.}
	\mauvaise{La boucle externe est une boucle infinie, le programme ne termine pas.}
	\end{reponses}
\end{question} \\
}


\element{StructuresTantQue}{
\begin{question}{TtQue05}
	Sélectionnez le code permettant d’obtenir le résultat suivant :
	\lstinputlisting{TtQue_05.py}
	\begin{reponses}	
	\bonne{\lstinputlisting{TtQue_05b.py}}
	\mauvaise{\lstinputlisting{TtQue_05a.py}}
	\mauvaise{\lstinputlisting{TtQue_05c.py}}
	\mauvaise{Aucune de ces propositions n'est exacte.}
	\end{reponses}
\end{question}\\
}

%\element{StructuresTantQue}{
%\begin{question}{}
%	\lstinputlisting{TtQue_04.py}
%\begin{multicols}{4}	
%	\begin{reponses}	
%	\bonne{}
%	\mauvaise{}
%	\mauvaise{}
%	\mauvaise{}
%	\end{reponses}
%	\end{multicols}
%\end{question}
%}
\element{StructuresConditionnelles}{
\begin{question}{}

\begin{multicols}{4}	
	\begin{reponses}	
	\bonne{}
	\mauvaise{}
	\mauvaise{}
	\mauvaise{}
	\end{reponses}
	\end{multicols}
\end{question}
}

\element{StructuresFor}{
\begin{question}{for01}  % genumsi nreveret
On souhaite écrire un programme affichant tous les entiers multiples de 3 entre 6 et 288 inclus.
Quel code est correct ?
	\begin{multicols}{2}	
	\begin{reponses}	
	\bonne{$\;$\lstinputlisting{For_01_d.py}}
	\mauvaise{$\;$\lstinputlisting{For_01_a.py}}
	\mauvaise{$\;$\lstinputlisting{For_01_b.py}}
	\mauvaise{$\;$\lstinputlisting{For_01_c.py}}
	\end{reponses}
	\end{multicols}
\end{question}\\}

\element{StructuresFor}{
\begin{question}{for02}  % genumsi nreveret
	On a saisi le code suivant :
	\lstinputlisting{For_02.py}
	Quelle est la valeur de \lstinline{a} après l’exécution du code ?
\begin{multicols}{4}	
	\begin{reponses}	
	\bonne{26}
	\mauvaise{18}
	\mauvaise{18.0}
	\mauvaise{26.0}
	\end{reponses}
	\end{multicols}
\end{question}\\}

\element{StructuresFor}{
\begin{question}{for03}% osupk8
\texttt{Pour i allant de 0 à 9} s'écrit :	
\begin{multicols}{2}
	\begin{reponses}	
	\bonne{\lstinline{for i in range(10) :}}
	\mauvaise{\lstinline{for i in range(8) :}}
	\mauvaise{\lstinline{for i in range(9) :}}
	\mauvaise{\lstinline{for i in range(11) :}}
	\end{reponses}
	\end{multicols}
\end{question}\\}

\element{StructuresFor}{
\begin{question}{for04}% osupk8
\texttt{pour i allant de 1 à 10} s'écrit :
\begin{multicols}{2}
	\begin{reponses}	
	\bonne{\lstinline{for i in range(1,11) :}}
	\mauvaise{\lstinline{for i in range(10) :}}
	\mauvaise{\lstinline{for i in range(1,10) :}}
	\mauvaise{\lstinline{for i in range(0,10) :}}
	\end{reponses}
	\end{multicols}
\end{question}\\}

\element{StructuresFor}{
\begin{question}{for05}% planchet.d
On a saisi le code suivant : 
	\lstinputlisting{For_05.py}
Qu'affiche le programme python ?
	\begin{reponses}	
	\bonne{4.}
	\mauvaise{5.}
	\mauvaise{0 puis 1 puis 2 puis 3 puis 4.}
	\mauvaise{0 puis 1 puis 2 puis 3 puis 4 puis 5.}
	\end{reponses}
\end{question}\\}

\element{StructuresFor}{
\begin{question}{for06}% sgenre
Qu'affiche le script suivant : 
	\lstinputlisting{For_06.py}
\begin{multicols}{4}
	\begin{reponses}	
	\bonne{15}
	\mauvaise{6}
	\mauvaise{20}
	\mauvaise{11}
	\end{reponses}
	\end{multicols}
\end{question}\\}

\element{StructuresFor}{
\begin{question}{for07}% 
\texttt{for i in range(5) :} signifie que \texttt{i} prend les valeurs :
	\begin{reponses}	
	\bonne{0, 1, 2, 3, 4.}
	\mauvaise{1, 2, 3, 4, 5.}
	\mauvaise{5, 4, 3, 2, 1.}
	\mauvaise{4, 3, 2, 1, 0.}
	\end{reponses}
\end{question}\\}

\element{StructuresFor}{
\begin{question}{for08}%
Quelles sont les valeurs prises successivement par la variable \texttt{i} dans la boucle for ci-dessous ?
	\lstinputlisting{For_08.py} 
\begin{multicols}{4}
	\begin{reponses}	
	\bonne{0, 1, 2.}
	\mauvaise{0, 1, 2, 3.}
	\mauvaise{1, 2, 3.}
	\mauvaise{1, 2, 3, 4.}
	\end{reponses}
	\end{multicols}
\end{question}\\}


%\element{StructuresFor}{
%\begin{question}{for}% 
%\begin{multicols}{4}
%	\begin{reponses}	
%	\bonne{}
%	\mauvaise{}
%	\mauvaise{}
%	\mauvaise{}
%	\end{reponses}
%	\end{multicols}
%\end{question}\\}
\element{Listes}{
\begin{question}{list01}  % genumsi nreveret
	Quelle est le résultat de :  \lstinline{[ (a,b) for a in range(3) for b in range(a) ]} ?
%\begin{multicols}{4}	
	\begin{reponses}	
	\bonne{\lstinline{[(1,0),(2,0),(2,1)]}}
	\mauvaise{\lstinline{[(1,0),(2,1),(2,1)]}}
	\mauvaise{\lstinline{[(1,0),(2,1),(3,2)]}}
	\mauvaise{\lstinline{[(0,0),(1,1),(2,2)]}}
	\end{reponses}
%\end{multicols}
\end{question}


\begin{question}{list02}  % genumsi osupk8

\begin{multicols}{4}
	Soit la liste :  \lstinline{liste_pays = ['France','Allemagne','Italie','Belgique','Pays Bas']}. Que renvoie l'instruction : \lstinline{'Belgique' in liste_pays}.	
	\begin{reponses}	
	\bonne{\lstinline{True}}
	\mauvaise{\lstinline{False}}
	\mauvaise{\lstinline{liste_pays}}
	\mauvaise{\lstinline{['France','Allemagne','Italie','Belgique','Pays Bas']}}
	\end{reponses}
	\end{multicols}
\end{question}


\begin{question}{list02}  % genumsi osupk8
	Soit la liste :  \lstinline{liste_pays = ['France','Allemagne','Italie','Belgique','Pays Bas']}. Que renvoie l'instruction : \lstinline{liste_pays[2]}.	
\begin{multicols}{4}	
	\begin{reponses}	
	\bonne{Italie}
	\mauvaise{France}
	\mauvaise{Allemagne}
	\mauvaise{Belgique}
	\end{reponses}
	\end{multicols}
\end{question}


\begin{question}{list01}  % genumsi nreveret

\begin{multicols}{4}	
	\begin{reponses}	
	\bonne{}
	\mauvaise{}
	\mauvaise{}
	\mauvaise{}
	\end{reponses}
	\end{multicols}
\end{question}
}
\element{FonctionsEtListes}{
\begin{question}{fonclist01}  % genumsi nreveret
Voici une fonction Python de recherche d'un maximum :
\lstinputlisting{FoncList_01.py}
Avec quelle précondition sur la liste \lstinline{t}, la postcondition `` \lstinline{m} est un élément maximum de la liste \lstinline{s}'' n'est-elle pas assurée ?
	\begin{reponses}
	\bonne{Tout élément de \lstinline{t} est un entier supérieur ou égal à -2.}
	\mauvaise{Tout élément de \lstinline{t} est un entier positif ou nul.}
	\mauvaise{Tout élément de \lstinline{t} est un entier supérieur ou égal à 11.}
	\mauvaise{Tout élément de \lstinline{t} est un entier strictement supérieur à -2.}
	\end{reponses}
\end{question}

\begin{question}{fonclist02}  % genumsi nreveret
On dispose d'un tableau d'entiers, ordonné en ordre croissant. On désire connaître le nombre de valeurs distinctes contenues dans ce tableau. 
Quelle est la fonction qui ne convient pas ? 
	\begin{reponses}	
	\bonne{$\;$\lstinputlisting{fonclist_02_d.py}}
	\mauvaise{$\;$\lstinputlisting{fonclist_02_a.py}}
	\mauvaise{$\;$ \lstinputlisting{fonclist_02_b.py}}
	\mauvaise{$\;$ \lstinputlisting{fonclist_02_c.py}}
	\end{reponses}
\end{question}

\begin{question}{}

\begin{multicols}{4}	
	\begin{reponses}	
	\bonne{}
	\mauvaise{}
	\mauvaise{}
	\mauvaise{}
	\end{reponses}
	\end{multicols}
\end{question}

}


\element{ListesComprehension}{
\begin{question}{LiCo01}% genumsi osupk8
Quelle est le résultat de :  \lstinline{[ (a,b) for a in range(3) for b in range(a) ]} ?
\begin{multicols}{4}	
	\begin{reponses}	
	\bonne{\lstinline{[(1,0),(2,0),(2,1)]}}
	\mauvaise{\lstinline{[(1,0),(2,1),(2,1)]}}
	\mauvaise{\lstinline{[(1,0),(2,1),(3,2)]}}
	\mauvaise{\lstinline{[(0,0),(1,1),(2,2)]}}
	\end{reponses}
	\end{multicols}
\end{question}

\begin{question}{lico}

\begin{multicols}{4}	
	\begin{reponses}	
	\bonne{}
	\mauvaise{}
	\mauvaise{}
	\mauvaise{}
	\end{reponses}
	\end{multicols}
\end{question}
}






\element{}{
\begin{question}{}

\begin{multicols}{4}	
	\begin{reponses}	
	\bonne{}
	\mauvaise{}
	\mauvaise{}
	\mauvaise{}
	\end{reponses}
	\end{multicols}
\end{question}
}
\melangegroupe{OperationsElementaires}
	\restituegroupe{OperationsElementaires}

	\melangegroupe{OperationsArithmetiques}
	\restituegroupe{OperationsArithmetiques}

\melangegroupe{Fonctions}
\restituegroupe{Fonctions}

	\melangegroupe{CodageEntiers}
	\restituegroupe{CodageEntiers}
	
	\melangegroupe{CodageEntiersRelatifs}
	\restituegroupe{CodageEntiersRelatifs}

	\melangegroupe{CodageHexa}
	\restituegroupe{CodageHexa}
	
	\melangegroupe{AlgebreBoole}
	\restituegroupe{AlgebreBoole}

	\melangegroupe{ChainesCaracteres}
	\restituegroupe{ChainesCaracteres}
	
	\melangegroupe{StructuresConditionnelles}
	\restituegroupe{StructuresConditionnelles}
	
	\melangegroupe{StructuresTantQue}
	\restituegroupe{StructuresTantQue}
	
	\melangegroupe{StructuresFor}
	\restituegroupe{StructuresFor}

	\melangegroupe{Listes}
	\restituegroupe{Listes}

	\melangegroupe{FonctionsEtListes}
	\restituegroupe{FonctionsEtListes}

	\melangegroupe{ListesComprehension}
	\restituegroupe{ListesComprehension}


	
\end{document}