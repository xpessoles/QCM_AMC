\element{OperationsElementaires}{
\begin{question}{OpEl 01}% genumsi nreveret
	On souhaite écrire un programme calculant le triple d'un nombre décimal et affichant le résultat. On a saisi le code suivant :
	\lstinputlisting{OpEl_01.py}
	Quel va être le résultat affiché ?
\begin{multicols}{4}
	\begin{reponses}	
	\bonne{555}
	\mauvaise{\lstinline{nombrenombrenombre}}
	\mauvaise{\lstinline{15}}
	\mauvaise{\lstinline{15.0}}
	\end{reponses}
	\end{multicols}
\end{question}

\begin{question}{OpEl 02} % nreveret
	On a saisi le code suivant : 
	\lstinputlisting{OpEl_02.py}
	Quelle instruction permet d’afficher le message \lstinline{1 + 1 = 2} ?
\begin{multicols}{2}	
	\begin{reponses}	
	\bonne{\lstinline{print(a + ' = ' + c)}}
	\mauvaise{\lstinline{print(a + ' = ' + d)}}
	\mauvaise{\lstinline{print(b + ' = ' + c)}}
	\mauvaise{\lstinline{print(b + ' = ' + d)}}
	\end{reponses}
	\end{multicols}
\end{question}

\begin{question}{OpEl 03} % nreveret
	On a saisi le code suivant : 
	\lstinputlisting{OpEl_03.py}
	Quelles sont les valeurs de a et b à la fin du programme ?
\begin{multicols}{4}	
	\begin{reponses}	
	\bonne{\lstinline{a = 5} et \lstinline{b = 8}}
	\mauvaise{\lstinline{a = 8} et \lstinline{b = 5}}
	\mauvaise{\lstinline{a = 8} et \lstinline{b = 13}}
	\mauvaise{\lstinline{a = 13} et \lstinline{b = 5}}
	\end{reponses}
	\end{multicols}
\end{question}


\begin{question}{OpEl 04} % osupk8
Que contient la variable \lstinline{a} si on execute ce script ?
\lstinputlisting{opel_04.py}
\begin{multicols}{4}	
	\begin{reponses}
	\bonne{-5.0}
	\mauvaise{5.0}
	\mauvaise{1.0}
	\mauvaise{-1.0}
	\end{reponses}
	\end{multicols}
\end{question}

\begin{question}{OpEl 05} % osupk8
Que contient la variable \lstinline{a} si on exécute ce script ?
	\lstinputlisting{OpEl_05.py}
\begin{multicols}{4}	
	\begin{reponses}	
	\bonne{16.0}
	\mauvaise{9.0}
	\mauvaise{10.0}
	\mauvaise{12.0}
	\end{reponses}
	\end{multicols}
\end{question}
\begin{question}{OpEl 06} % osupk8
Que contiennent les variables \lstinline{a} et \lstinline{b} si on execute ce script ?
	\lstinputlisting{OpEl_06.py}
\begin{multicols}{4}	
	\begin{reponses}
	\bonne{5.0 et 7.0}
	\mauvaise{5.0 et 5.0}
	\mauvaise{7.0 et 5.0}
	\mauvaise{7.0 et 7.0}
	\end{reponses}
	\end{multicols}
\end{question}
\begin{question}{OpEl 07}%% osupk8
Que taper en Python pour obtenir $3^8$ ?
\begin{multicols}{4}	
	\begin{reponses}	
	\bonne{\lstinline{3**8}}
	\mauvaise{\lstinline{3^8}}
	\mauvaise{\lstinline{3*8}}
	\mauvaise{\lstinline{3&8}}
	\end{reponses}
	\end{multicols}
\end{question}



\begin{question}{OpEl 08}%% planchet.d
On a saisi le code suivant : \lstinline{a = '8'} puis \lstinline{b = 5} et \lstinline{a + b}. Que retourne ce programme ?
\begin{multicols}{4}	
	\begin{reponses}	
	\bonne{TypeError : must be str, not int.}
	\mauvaise{'13'}
	\mauvaise{False}
	\mauvaise{13}
	\end{reponses}
	\end{multicols}
\end{question}


\begin{question}{OpEl 09}%% planchet.d
On souhaite écrire un programme calculant le triple d'un nombre décimal et affichant le résultat. On a saisi le code suivant : \lstinline{nombre = `5'} puis \lstinline{triple = nombre * 3}. Quel va être le résultat affiché en saisissant \lstinline{print(triple)}?
\begin{multicols}{4}	
	\begin{reponses}	
	\bonne{555}
	\mauvaise{15}
	\mauvaise{15.0}
	\mauvaise{nombrenombrenombre}
	\end{reponses}
	\end{multicols}
\end{question}



\begin{question}{OpEl 10}%% sgenre
En python, que fait l'instruction suivante ? \lstinline{#print(a,b)}
\begin{multicols}{4}	
	\begin{reponses}	
	\bonne{Elle ne fait rien.}
	\mauvaise{Elle affiche le texte 'a,b'.} 
	\mauvaise{Elle affiche les valeurs de a et b.}
	\mauvaise{Elle génère une erreur.}
	\end{reponses}
	\end{multicols}
\end{question}

\begin{question}{OpEl 11}% sgenre
En python, combien vaut : \lstinline{12%5} ?
\begin{multicols}{4}	
	\begin{reponses}	
	\bonne{3}
	\mauvaise{1}
	\mauvaise{3}
	\mauvaise{Ce calcul génère une erreur de calcul.}
	\end{reponses}
	\end{multicols}
\end{question}

\begin{question}{OpEl}% sgenre

\begin{multicols}{4}	
	\begin{reponses}	
	\bonne{}
	\mauvaise{}
	\mauvaise{}
	\mauvaise{}
	\end{reponses}
	\end{multicols}
\end{question}



}