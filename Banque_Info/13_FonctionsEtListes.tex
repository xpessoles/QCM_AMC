\element{FonctionsEtListes}{
\begin{question}{fonclist01}  % genumsi nreveret
Voici une fonction Python de recherche d'un maximum :
\lstinputlisting{FoncList_01.py}
Avec quelle précondition sur la liste \lstinline{t}, la postcondition `` \lstinline{m} est un élément maximum de la liste \lstinline{s}'' n'est-elle pas assurée ?
	\begin{reponses}
	\bonne{Tout élément de \lstinline{t} est un entier supérieur ou égal à -2.}
	\mauvaise{Tout élément de \lstinline{t} est un entier positif ou nul.}
	\mauvaise{Tout élément de \lstinline{t} est un entier supérieur ou égal à 11.}
	\mauvaise{Tout élément de \lstinline{t} est un entier strictement supérieur à -2.}
	\end{reponses}
\end{question}}

\element{FonctionsEtListes}{
\begin{question}{fonclist02}  % genumsi nreveret
On dispose d'un tableau d'entiers, ordonné en ordre croissant. On désire connaître le nombre de valeurs distinctes contenues dans ce tableau. 
Quelle est la fonction qui ne convient pas ? 
	\begin{reponses}	
	\bonne{$\;$\lstinputlisting{fonclist_02_d.py}}
	\mauvaise{$\;$\lstinputlisting{fonclist_02_a.py}}
	\mauvaise{$\;$ \lstinputlisting{fonclist_02_b.py}}
	\mauvaise{$\;$ \lstinputlisting{fonclist_02_c.py}}
	\end{reponses}
\end{question}}

%\begin{question}{}
%\begin{multicols}{4}	
%	\begin{reponses}	
%	\bonne{}
%	\mauvaise{}
%	\mauvaise{}
%	\mauvaise{}
%	\end{reponses}
%	\end{multicols}
%\end{question}
%
%}
%
