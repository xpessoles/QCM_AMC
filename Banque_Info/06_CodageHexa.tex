\element{CodageHexa}{
\begin{question}{hexa 01} % genumsi nreveret
	Convertir  la valeur décimale 195 en hexadécimal.
	\begin{multicols}{4}
	\begin{reponses}
	\bonne{C3}
	\mauvaise{A5}
	\mauvaise{B9}
	\mauvaise{C9}
	\end{reponses}
	\end{multicols}
\end{question}\\}

\element{CodageHexa}{
\begin{question}{hexa 02}% genumsi OSUPK8
	Donner l'écriture hexadécimale du nombre binaire 1001011.
	\begin{multicols}{4}
	\begin{reponses}	
	\bonne{4B}
	\mauvaise{3D}
	\mauvaise{49}
	\mauvaise{5B}
	\end{reponses}
	\end{multicols}
\end{question}\\}

\element{CodageHexa}{
\begin{question}{hexa 03}% genumsi OSUPK8
	Donner l'écriture hexadécimale du nombre binaire 110101.
	\begin{multicols}{4}
	\begin{reponses}	
	\bonne{35}
	\mauvaise{6B}
	\mauvaise{65}
	\mauvaise{56}
	\end{reponses}
	\end{multicols}
\end{question}\\}

\element{CodageHexa}{
\begin{question}{hexa 04}% genumsi OSUPK8
	Donner l'écriture binaire du nombre hexadécimal 6E.
	\begin{multicols}{4}
	\begin{reponses}	
	\bonne{01101110}
	\mauvaise{01110110}
	\mauvaise{01101101}
	\mauvaise{01110010}
	\end{reponses}
	\end{multicols}
\end{question}\\}

\element{CodageHexa}{
\begin{question}{hexa 05}% genumsi OSUPK8
	Donner l'écriture binaire du nombre hexadécimal B5.
	\begin{multicols}{4}
	\begin{reponses}	
	\bonne{10110101}
	\mauvaise{10110111}
	\mauvaise{00110101}
	\mauvaise{10111101}
	\end{reponses}
	\end{multicols}
\end{question}\\}

\element{CodageHexa}{
\begin{question}{hexa 06}% genumsi OSUPK8
	Quelle est la représentation binaire du nombre $5D_{16}$ ?
	\begin{multicols}{4}
	\begin{reponses}	
	\bonne{01011101}
	\mauvaise{01101101}
	\mauvaise{10101101}
	\mauvaise{01011110}
	\end{reponses}
	\end{multicols}
\end{question}\\}

\element{CodageHexa}{
\begin{question}{hexa 07}% genumsi OSUPK8
	Quelle est la valeur hexadécimale de l'entier binaire 10110110 ?
	\begin{multicols}{4}
	\begin{reponses}
	\bonne{B6}
	\mauvaise{C4}
	\mauvaise{B8}
	\mauvaise{C6}
	\end{reponses}
	\end{multicols}
\end{question}\\}

\element{CodageHexa}{
\begin{question}{hexa 08}% genumsi OSUPK8
	\begin{multicols}{4}
	\begin{reponses}	
	\bonne{}
	\mauvaise{}
	\mauvaise{}
	\mauvaise{}
	\end{reponses}
	\end{multicols}
\end{question}\\}

\element{CodageHexa}{
\begin{question}{hexa }% genumsi OSUPK8
	\begin{multicols}{4}
	\begin{reponses}	
	\bonne{}
	\mauvaise{}
	\mauvaise{}
	\mauvaise{}
	\end{reponses}
	\end{multicols}
\end{question}\\}

%\element{CodageHexa}{
%\begin{question}{hexa }% genumsi OSUPK8
%	\begin{multicols}{4}
%	\begin{reponses}	
%	\bonne{}
%	\mauvaise{}
%	\mauvaise{}
%	\mauvaise{}
%	\end{reponses}
%	\end{multicols}
%\end{question}
%}
