\documentclass[a4paper]{article}

% bloc : évite un changement de page
% completemulti : ajoute une case : aucune bonne réponse
%\newcommand{\repRel}{../..}
%\input{\repRel/Style/packages}
%%\input{\repRel/Style/new_style}
%\input{\repRel/Style/macros_SII}
%%\input{\repRel/Style/environment}

\usepackage[francais,bloc,completemulti,ensemble]{automultiplechoice}
%\usepackage[francais,bloc,completemulti]{automultiplechoice} 
\usepackage{amsmath}
% ensemble : feuille de questions et feuille de réponse séparées

\usepackage{multicol}

%% Pour le python %%
\usepackage{listingsutf8}

\lstset{language=Python,
  inputencoding=utf8/latin1,
  breaklines=true,
  basicstyle=\ttfamily\small,
  keywordstyle=\bfseries\color{green!40!black},
  commentstyle=\itshape\color{purple!40!black},
  identifierstyle=\color{blue},
  stringstyle=\color{orange},
  upquote = true,
  columns=fullflexible,
  backgroundcolor=\color{gray!10},frame=leftline,rulecolor=\color{gray}}  
  
\definecolor{mygreen}{rgb}{0,0.6,0}

\lstset{
     literate=%
         {é}{{\'e}}1    
         {è}{{\`e}}1    
         {ê}{{\^e}}1    
         {à}{{\`a}}1
         {â}{{\^a}}1		 
         {ô}{{\^o}}1    
         {ù}{{\`u}}1    
         {î}{{\^i}}1    
}
\lstset{inputpath=code}

\usepackage{siunitx}
%\usepackage[utf8x]{inputenc}
%\usepackage[T1]{fontenc}

\begin{document}
\AMCrandomseed{233893}
\setdefaultgroupmode{withoutreplacement} % Mélange des questions groupées

%\graphicspath{{../Banque_Info/02_FTBO/images/}}
\lstset{inputpath=../../Banque_Info/code}
\element{OperationsElementaires}{
\begin{question}{OpEl 01}% genumsi nreveret
	On souhaite écrire un programme calculant le triple d'un nombre décimal et affichant le résultat. On a saisi le code suivant :
	\lstinputlisting{OpEl_01.py}
	Quel va être le résultat affiché ?
	\begin{reponses}	
	\bonne{555}
	\mauvaise{\lstinline{nombrenombrenombre}}
	\mauvaise{\lstinline{15}}
	\mauvaise{\lstinline{15.0}}
	\end{reponses}
\end{question}\\}

\element{OperationsElementaires}{
\begin{question}{OpEl 02} % nreveret
	On a saisi le code suivant : 
	\lstinputlisting{OpEl_02.py}
	Quelle instruction permet d’afficher le message \lstinline{1 + 1 = 2} ?
\begin{multicols}{2}	
	\begin{reponses}	
	\bonne{\lstinline{print(a + ' = ' + c)}}
	\mauvaise{\lstinline{print(a + ' = ' + d)}}
	\mauvaise{\lstinline{print(b + ' = ' + c)}}
	\mauvaise{\lstinline{print(b + ' = ' + d)}}
	\end{reponses}
	\end{multicols}
\end{question}\\}

\element{OperationsElementaires}{
\begin{question}{OpEl 03} % nreveret
	On a saisi le code suivant : 
	\lstinputlisting{OpEl_03.py}
	Quelles sont les valeurs de a et b à la fin du programme ?
\begin{multicols}{4}	
	\begin{reponses}	
	\bonne{\lstinline{a = 5} et \lstinline{b = 8}}
	\mauvaise{\lstinline{a = 8} et \lstinline{b = 5}}
	\mauvaise{\lstinline{a = 8} et \lstinline{b = 13}}
	\mauvaise{\lstinline{a = 13} et \lstinline{b = 5}}
	\end{reponses}
	\end{multicols}
\end{question}\\}

\element{OperationsElementaires}{
\begin{question}{OpEl 04} % osupk8
Que contient la variable \lstinline{a} si on execute ce script ?
\lstinputlisting{opel_04.py}
\begin{multicols}{4}	
	\begin{reponses}
	\bonne{-5.0}
	\mauvaise{5.0}
	\mauvaise{1.0}
	\mauvaise{-1.0}
	\end{reponses}
	\end{multicols}
\end{question}\\}

\element{OperationsElementaires}{
\begin{question}{OpEl 05} % osupk8
Que contient la variable \lstinline{a} si on exécute ce script ?
	\lstinputlisting{OpEl_05.py}
\begin{multicols}{4}	
	\begin{reponses}	
	\bonne{16.0}
	\mauvaise{9.0}
	\mauvaise{10.0}
	\mauvaise{12.0}
	\end{reponses}
	\end{multicols}
\end{question}\\}

\element{OperationsElementaires}{
\begin{question}{OpEl 06} % osupk8
Que contiennent les variables \lstinline{a} et \lstinline{b} si on execute ce script ?
	\lstinputlisting{OpEl_06.py}
\begin{multicols}{4}	
	\begin{reponses}
	\bonne{5.0 et 7.0}
	\mauvaise{5.0 et 5.0}
	\mauvaise{7.0 et 5.0}
	\mauvaise{7.0 et 7.0}
	\end{reponses}
	\end{multicols}
\end{question}\\}

\element{OperationsElementaires}{
\begin{question}{OpEl 07}%% osupk8
Que taper en Python pour obtenir $3^8$ ?
\begin{multicols}{4}	
	\begin{reponses}	
	\bonne{\lstinline{3**8}}
	\mauvaise{\lstinline{3^8}}
	\mauvaise{\lstinline{3*8}}
	\mauvaise{\lstinline{3&8}}
	\end{reponses}
	\end{multicols}
\end{question}\\}

\element{OperationsElementaires}{
\begin{question}{OpEl 08}%% planchet.d
On a saisi le code suivant : \lstinline{a = '8'} puis \lstinline{b = 5} et \lstinline{a + b}. Que retourne ce programme~?
	\begin{reponses}	
	\bonne{TypeError : must be str, not int.}
	\mauvaise{'13'}
	\mauvaise{False}
	\mauvaise{13}
	\end{reponses}
\end{question}\\}

\element{OperationsElementaires}{
\begin{question}{OpEl 09}%% planchet.d
On souhaite écrire un programme calculant le triple d'un nombre décimal et affichant le résultat. On a saisi le code suivant : \lstinline{nombre = '5'} puis \lstinline{triple = nombre * 3}. Quel va être le résultat affiché en saisissant \lstinline{print(triple)}?

	\begin{reponses}	
	\bonne{555}
	\mauvaise{15}
	\mauvaise{15.0}
	\mauvaise{nombrenombrenombre}
	\end{reponses}

\end{question}\\}

\element{OperationsElementaires}{
\begin{question}{OpEl 10}%% sgenre
En python, que fait l'instruction suivante ? \lstinline{#print(a,b)}
	\begin{reponses}	
	\bonne{Elle ne fait rien.}
	\mauvaise{Elle affiche le texte 'a,b'.} 
	\mauvaise{Elle affiche les valeurs de a et b.}
	\mauvaise{Elle génère une erreur.}
	\end{reponses}
\end{question}\\}

\element{OperationsElementaires}{
\begin{question}{OpEl 11}% sgenre
En python, combien vaut : \texttt{12\%5} ?
	\begin{reponses}	
	\bonne{3}
	\mauvaise{1}
	\mauvaise{3}
	\mauvaise{Ce calcul génère une erreur de calcul.}
	\end{reponses}
\end{question}\\}
%
%\element{OperationsElementaires}{
%\begin{question}{OpEl}% sgenre
%
%\begin{multicols}{4}	
%	\begin{reponses}	
%	\bonne{}
%	\mauvaise{}
%	\mauvaise{}
%	\mauvaise{}
%	\end{reponses}
%	\end{multicols}
%\end{question}
%\\}
\element{StructuresConditionnelles}{
\begin{question}{if 01}%sgenre
On définit la fonction mystère suivante : 
	\lstinputlisting{if_01.py}
Quelle est la valeur de mystere(10) ?
\begin{multicols}{4}	
	\begin{reponses}	
	\bonne{'B'}
	\mauvaise{'A'}
	\mauvaise{'C'}
	\mauvaise{'D'}
	\end{reponses}
	\end{multicols}
\end{question}\\}



\element{StructuresConditionnelles}{
\begin{question}{if 02}%trevien
Quel est le résultat de ce code ?
	\lstinputlisting{if_02.py}

\begin{multicols}{4}	
	\begin{reponses}	
	\bonne{'3'}
	\mauvaise{'5'}
	\mauvaise{'7'}
	\mauvaise{'37'}
	\end{reponses}
	\end{multicols}
\end{question}\\}

\element{StructuresConditionnelles}{
\begin{question}{if 03}%trevien
Quel est le résultat de ce code ?
	\lstinputlisting{if_03.py}
	\begin{reponses}	
	\bonne{20}
	\mauvaise{\texttt{File 'input', line 1, in <module>if not true:NameError: name 'true' is not defined}}
	\mauvaise{30}
	\mauvaise{Aucune de ces proposition n'est exacte.}
	\end{reponses}
\end{question}\\}

\element{StructuresConditionnelles}{
\begin{question}{if 04}%trevien
Quel est le résultat de ce code ?
	\lstinputlisting{if_04.py}

	\begin{reponses}	
	\bonne{Non}
	\mauvaise{OK}
	\mauvaise{\texttt{File 'input', line 5 elif x>y SyntaxError: invalid syntax.}}
	\mauvaise{Aucune de ces propositions n'est exacte.}
	\end{reponses}
\end{question}\\}

\element{StructuresConditionnelles}{
\begin{question}{if 05}%trevien
On souhaite définir une fonction qui compare la longueur de deux chaînes de caractères et renvoie la plus courte. Pour cela, il faudrait compléter le code suivant :
	\lstinputlisting{if_05.py}
	\begin{reponses}	
	\bonne{\texttt{return x} et \texttt{return y}}
	\mauvaise{\texttt{print(x)} et \texttt{print(y)}}
	\mauvaise{\texttt{print(x)} et \texttt{return(y)}}
	\mauvaise{\texttt{return x} et \texttt{print(y)}}
	\end{reponses}
\end{question}\\}


%\element{StructuresConditionnelles}{
%\begin{question}{if 0n}%trevien
%	\lstinputlisting{if_01.py}
%
%\begin{multicols}{4}	
%	\begin{reponses}	
%	\bonne{}
%	\mauvaise{}
%	\mauvaise{}
%	\mauvaise{}
%	\end{reponses}
%	\end{multicols}
%\end{question}
%}

\element{StructuresFor}{
\begin{question}{for01}  % genumsi nreveret
On souhaite écrire un programme affichant tous les entiers multiples de 3 entre 6 et 288 inclus.
Quel code est correct ?
	\begin{multicols}{2}	
	\begin{reponses}	
	\bonne{$\;$\lstinputlisting{For_01_d.py}}
	\mauvaise{$\;$\lstinputlisting{For_01_a.py}}
	\mauvaise{$\;$\lstinputlisting{For_01_b.py}}
	\mauvaise{$\;$\lstinputlisting{For_01_c.py}}
	\end{reponses}
	\end{multicols}
\end{question}\\}

\element{StructuresFor}{
\begin{question}{for02}  % genumsi nreveret
	On a saisi le code suivant :
	\lstinputlisting{For_02.py}
	Quelle est la valeur de \lstinline{a} après l’exécution du code ?
\begin{multicols}{4}	
	\begin{reponses}	
	\bonne{26}
	\mauvaise{18}
	\mauvaise{18.0}
	\mauvaise{26.0}
	\end{reponses}
	\end{multicols}
\end{question}\\}

\element{StructuresFor}{
\begin{question}{for03}% osupk8
\texttt{Pour i allant de 0 à 9} s'écrit :	
\begin{multicols}{2}
	\begin{reponses}	
	\bonne{\lstinline{for i in range(10) :}}
	\mauvaise{\lstinline{for i in range(8) :}}
	\mauvaise{\lstinline{for i in range(9) :}}
	\mauvaise{\lstinline{for i in range(11) :}}
	\end{reponses}
	\end{multicols}
\end{question}\\}

\element{StructuresFor}{
\begin{question}{for04}% osupk8
\texttt{pour i allant de 1 à 10} s'écrit :
\begin{multicols}{2}
	\begin{reponses}	
	\bonne{\lstinline{for i in range(1,11) :}}
	\mauvaise{\lstinline{for i in range(10) :}}
	\mauvaise{\lstinline{for i in range(1,10) :}}
	\mauvaise{\lstinline{for i in range(0,10) :}}
	\end{reponses}
	\end{multicols}
\end{question}\\}

\element{StructuresFor}{
\begin{question}{for05}% planchet.d
On a saisi le code suivant : 
	\lstinputlisting{For_05.py}
Qu'affiche le programme python ?
	\begin{reponses}	
	\bonne{4.}
	\mauvaise{5.}
	\mauvaise{0 puis 1 puis 2 puis 3 puis 4.}
	\mauvaise{0 puis 1 puis 2 puis 3 puis 4 puis 5.}
	\end{reponses}
\end{question}\\}

\element{StructuresFor}{
\begin{question}{for06}% sgenre
Qu'affiche le script suivant : 
	\lstinputlisting{For_06.py}
\begin{multicols}{4}
	\begin{reponses}	
	\bonne{15}
	\mauvaise{6}
	\mauvaise{20}
	\mauvaise{11}
	\end{reponses}
	\end{multicols}
\end{question}\\}

\element{StructuresFor}{
\begin{question}{for07}% 
\texttt{for i in range(5) :} signifie que \texttt{i} prend les valeurs :
	\begin{reponses}	
	\bonne{0, 1, 2, 3, 4.}
	\mauvaise{1, 2, 3, 4, 5.}
	\mauvaise{5, 4, 3, 2, 1.}
	\mauvaise{4, 3, 2, 1, 0.}
	\end{reponses}
\end{question}\\}

\element{StructuresFor}{
\begin{question}{for08}%
Quelles sont les valeurs prises successivement par la variable \texttt{i} dans la boucle for ci-dessous ?
	\lstinputlisting{For_08.py} 
\begin{multicols}{4}
	\begin{reponses}	
	\bonne{0, 1, 2.}
	\mauvaise{0, 1, 2, 3.}
	\mauvaise{1, 2, 3.}
	\mauvaise{1, 2, 3, 4.}
	\end{reponses}
	\end{multicols}
\end{question}\\}


%\element{StructuresFor}{
%\begin{question}{for}% 
%\begin{multicols}{4}
%	\begin{reponses}	
%	\bonne{}
%	\mauvaise{}
%	\mauvaise{}
%	\mauvaise{}
%	\end{reponses}
%	\end{multicols}
%\end{question}\\}



\exemplaire{1}{
% Entetes sujet
\noindent{\bf PTSI 2 -- QCM 01}

\vspace*{.5cm}
% Questions
\restituegroupe[2]{OperationsElementaires}
\restituegroupe[2]{StructuresConditionnelles}
\restituegroupe[2]{StructuresFor}

\AMCcleardoublepage


\AMCdebutFormulaire
% ENtetes
{\large\bf Feuille de réponses :}
%
\vspace{.5cm}

\begin{minipage}[c]{.45\linewidth}
{\large\bf Noircir votre numéro personnel.}

\vspace{.5cm}

\AMCcodeGridInt[h]{etu}{2}
\end{minipage}
\hfill
\begin{minipage}[c]{.45\linewidth}
\champnom{\fbox{
\begin{minipage}{.9\linewidth}
Nom et prénom :

\vspace*{.5cm}\dotfill

\vspace*{.5cm}\dotfill

\vspace*{1mm}
\end{minipage}
}}
\end{minipage}

%%%%%%%

\vspace*{.5cm}
Pour répondre aux questions \textbf{noircir consciencieusement} la réponse sélectionnée. 
\vspace*{.5cm}

\formulaire

% 
}

\end{document}
